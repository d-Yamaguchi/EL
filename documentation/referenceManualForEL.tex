\AtBeginDvi{\special{pdf: pagesize width 182truemm height 257truemm}} 
\documentclass[openany,b5j,11pt,english]{jsbook}
\usepackage{dnjbook,rsfs}
\setlength{\topmargin}{-2truecm}
\setlength{\textheight}{22truecm}
\def\EL{\reflectbox{E}\kern-.3em\hbox{L}}
\title{\underline{{\huge $\;\:$\textit{The Reference Manual for} \EL$\;\:$}}\\Version 3.141}
\author{The \EL \;Development Team}
\date{\today}
\lstset{
  showspaces=false,
  showtabs=false,
  breaklines=true,
  showstringspaces=false,
  breakatwhitespace=true,
  commentstyle=\color{pgreen},
  keywordstyle=\color{pblue},
  stringstyle=\color{pred},
  basicstyle={\small,\ttfamily},%
  moredelim=[il][\textcolor{pgrey}]{$$},
  moredelim=[is][\textcolor{pgrey}]{\%\%}{\%\%},
  numbers=left,stepnumber=1,numberstyle=\footnotesize,
  frame=tRBl,framesep=5pt,
  classoffset=1,%
}
\begin{document}
\maketitle
\frontmatter
\setcounter{page}{0}
\thispagestyle{empty}
$\;$\\
\newpage
\tableofcontents
%\listoffigures
%\listoftables
\mainmatter
\chapter{Language Guide}
%EL特徴のあらまし
\section{types}
%提供している型について
\EL\;provides fundamental types, including \texttt{Int} for integers, \texttt{Double} for floating-point values, \texttt{String} for textual data, and \texttt{Bool} for Boolean value.
\EL\;also provides Tuple and List as described in \S 1.3, \S1.4.
\section{constants}
%定数は使えますが,変数は使えません(参照透過性)
You cannot use variable. Alternatively, \EL\;provides constant. The value of a constant cannot be changed once it is set in the future. This constraint ensures referential transparency.
\section{list}
%リストには3通りの記述方法があります
In \EL, lists are a homogenous data structure. 
It stores several elements of the same type.
List can be written in three different ways (Enumeration, Range , List comprehension).
It should be noted that you don't have to declare the list with its type.
\section{tuple}
%タプルが使えます
\EL\;provide tuple type. You can bound some values with tuple and handle as an unit value.
\section{function}
%ELの関数の特徴は必ず値を返すことです.
In \EL, you must define function to return just a value.
Besides, you must declare types of each arguments and type of the return value. 
Using constants locally, You could declare them in \texttt{where} block below the function.
\section{control flow}
%パターンマッチと再帰で
\EL\;provide only pattern-mach for control flow. Any conditional balancing you need is constructed with pattern-mach and several recursive call.
\chapter{Usage}
\section{declare constants}
\subsection{declare constant}
In this section, we describe the way to declare a constant.
Here is the sample code to decleare $n = 2$.
\lstinputlisting[caption=declare a constant \texttt{n},label=constDecl]
{../test/constDecl.txt}
In the first line, declare type of the constant named \texttt{n}. This time \texttt{n} is \texttt{Int} type.
Then, describe the value of \texttt{n} in the right hand of "\texttt{=}" in the second line. 
\subsection{declare list}
In this section, we describe the way to declare a list.
Here is a sample code to declare a list, which holds \texttt{Int} datas \texttt{\{1,2,3,4,5\}}. 
\lstinputlisting[caption=declare a list \texttt{listA},label=listDecl1]
{../test/listDecl1.txt}
You could declare a list without type declaration.
You need only single line to declare a list like that example.\\

We could also describe same list with \textsf{Range} expression like this.
\lstinputlisting[caption=declare a list \texttt{listA} with \texttt{Range},label=listDecl2]
{../test/listDecl2.txt}

\section{declare function}
\lstinputlisting[caption=function $fibo(n)$,label=fibo]
{../test/fibo.txt}
\section{pattern match}
\section{Input and Output}
\section{sample code}
Here is the sample code, which returns a fibonacci number.
\lstinputlisting[caption=sample code,label=sample]
{../test/test.txt}

\end{document}