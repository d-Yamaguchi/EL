\AtBeginDvi{\special{pdf: pagesize width 182truemm height 257truemm}} 
\documentclass[openany,b5j,11pt,english]{jsbook}
\usepackage{dnjbook,rsfs}
\setlength{\topmargin}{-2truecm}
\setlength{\textheight}{22truecm}
\def\EL{\reflectbox{E}\kern-.3em\hbox{L}}
\title{\underline{{\huge $\;\:$\textit{The Reference Manual for} \EL$\;\:$}}\\Version 3.14}
\author{The \EL \;Development Team}
\date{\today}
\lstset{
  showspaces=false,
  showtabs=false,
  breaklines=true,
  showstringspaces=false,
  breakatwhitespace=true,
  commentstyle=\color{pgreen},
  keywordstyle=\color{pblue},
  stringstyle=\color{pred},
  basicstyle={\small,\ttfamily},%
  moredelim=[il][\textcolor{pgrey}]{$$},
  moredelim=[is][\textcolor{pgrey}]{\%\%}{\%\%},
  numbers=left,stepnumber=1,numberstyle=\footnotesize,
  frame=tRBl,framesep=5pt,
  classoffset=1,%
}
\begin{document}
\maketitle
\frontmatter
\setcounter{page}{0}
\thispagestyle{empty}
$\;$\\
\newpage
\tableofcontents
%\listoffigures
%\listoftables
\mainmatter
\chapter{Language Guide}
\section{the basics}
%EL特徴のあらまし
\section{types}
%提供している型について
\section{constants}
%定数は使えますが,変数は使えません(参照透過性)
\section{list}
%リストには3通りの記述方法があります
\section{tuple}
%タプルが使えます
\section{function}
%ELの関数の特徴は必ず値を返すことです.
\section{control flow}
%パターンマッチと再帰で
\chapter{Usage}
\section{declare constants}
\section{declare function}
\lstinputlisting[caption=function $fibo(n)$,label=fibo]
{../test/fibo.txt}
\section{pattern match}
\section{Input and Output}
\section{sample code}
Here is the sample code, which returns a fibonacci number.
\lstinputlisting[caption=sample code,label=sample]
{../test/test.txt}

\end{document}